
\documentclass{resume} % Use the custom resume.cls style

\usepackage[left=0.75in,top=0.6in,right=0.75in,bottom=0.6in]{geometry} % Document margins
\usepackage{enumitem}
\newcommand{\tab}[1]{\hspace{.2667\textwidth}\rlap{#1}}
\newcommand{\itab}[1]{\hspace{0em}\rlap{#1}}
\name{Qichen Dong} % Your name
\address{207 Chapel Street, Salford, UK, M3 5PH} % Your address
\address{(+44) 742-250-7895 \\ qichen.dong@cern.ch \\ linkedin.com/in/qichendong} % Your phone number and email
\begin{document}

    \begin{rSection}{Education}
        \begin{rSubsection}{University of Manchester}{September 2020 - September 2024}{PhD Student in High Energy Physics}{Manchester, UK}
            \item Expected September 2024 
        \end{rSubsection}
        \begin{rSubsection}{University of Manchester}{September 2018 - June 2020}{Master of Physics}{Manchester, UK}
            \item First Class Honours
            \item Overall GPA: 74.4
            \item Ranked 30th out of 150 students in the final year.
        \end{rSubsection}
        \begin{rSubsection}{Shandong University}{September 2015 - June 2018}{Bachelor of Science in Physics}{Jinan, CN}
            \item Overall GPA: 83.3
        \end{rSubsection}
    \end{rSection}

    \begin{rSection}{Research Experience}
        \begin{rSubsection}{University of Manchester}{Sep 2020 - September 2024}{PhD Project}{Manchester, UK}
            \item   Project supervised by Prof. Terry Wyatt.
            \item   Harnessing the power of AI to analyse massive datasets generated by the ATLAS experiments. 
            \item   Proposed, developed, implemented, and tested improved methods to identify the highly boosted pair production of the heaviest $\tau$ leptons. Algorithm runs in Tier-0 ATLAS data processing system.
            \item   Techniques benchmarked with data, achieving a three- to four-fold improvement in the signal-to-background ratio. 
            \item   Leading an analysis searching for resonant production of the Higgs boson pairs in the $bb\tau\tau$ channel.
            \item   At least two papers are scheduled to be published for the benefit of the ATLAS collaboration, in which I will be the first author.
        \end{rSubsection}
        \begin{rSubsection}{ATLAS experiment, CERN}{April 2022 - August 2022}{Developer and Maintainer}{Geneva, CH}
            \item Long-term-attached PhD student.
            \item Developer and reviewer of the ATLAS offline software.
            \item Senior shifter in the ATLAS software merge-requests review team.
        \end{rSubsection}
        \begin{rSubsection}{University of Manchester}{September 2019 - June 2020}{Master Project}{Manchester, UK}
            \item   Project supervised by Prof. Andrew Pilkington. Achieved 81\%.
            \item   Searching for extra source of CP violation which contributes to the large matter anti-matter asymmetry in the universe.
            \item   The project measured the fiducial cross-section of the Vector Boson Scattering (VBS) processes.
            \item   One of the first to set preliminary limits on the Standard Model Effective Field Theory parameters in VBS processes.
            \item   Results presented to the ATLAS Collaboration. 
        \end{rSubsection}
    \end{rSection}

    \newpage

    \begin{rSection}{Work Experience}
        \begin{rSubsection}{Qube Research and Technology}{August 2022 - February 2023}{Staffed Intern Quantitative Researcher}{London, UK}
            \item Salaried internship.
            \item Developing and optimising database infrastructure for real-time financial market data, enhancing data retrieval and storage efficiency.
            \item Collaborated with a team to address critical NLP-related challenges in the finance industry leveraging AI.
        \end{rSubsection}
    \end{rSection}


    \begin{rSection}{Additional Experience}
        \begin{rSubsection}{University of Manchester}{Sep 2020 - Present}{Teaching Assistant}{Manchester, UK}
            \item   Demonstrating the C++ / Python laboratories for year-2 and year-3 undergraduate physics students.
            \item   Sharing fundamental programming skills with the next-gen programmers.
        \end{rSubsection}
        \begin{rSubsection}{Remote area in Sichuan}{June 2016 - September 2017}{Volunteer Primary School Teacher}{Sichuan, CN}
            \item   Helping primary school students who live in remote area in Sichuan province during summer break.
            \item   The team helped hundreds of children.
            \item   Local tobacco farming industry which can potentially reduce poverty in local area was investigated.
        \end{rSubsection}
        \begin{rSubsection}{Shandong University}{September 2015 - June 2017}{Lead of Student Union}{Jinan, CN}
            \item   Responsible for organising voluntary works hosted by the university.
            \item   Managing a team of 20 SU undergraduate representatives.
        \end{rSubsection}
    \end{rSection}
    
    \begin{rSection}{Skills and Interests}
        \begin{tabular}{ @{} >{\bfseries}l @{\hspace{6ex}} l }
            Programming & Proficient in programming with C/C++ and Python \\
            Teamwork    & Strong communication skill in highly collaborative environments \\ 
            Languages   & Chinese (Native), English (Professional) \\ 
            Interests   & Graphic design, Accelarated / Distributed computing \\  
        \end{tabular}
      \end{rSection}
\end{document}