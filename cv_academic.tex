
\documentclass{resume} % Use the custom resume.cls style

\usepackage[left=0.75in,top=0.6in,right=0.75in,bottom=0.6in]{geometry} % Document margins
\usepackage{enumitem}
\newcommand{\tab}[1]{\hspace{.2667\textwidth}\rlap{#1}}
\newcommand{\itab}[1]{\hspace{0em}\rlap{#1}}
\name{Qichen Dong} % Your name
\address{44 Whitworth Street, Manchester, M1 3AJ, UK} % Your address
\address{(+44) 742-250-7895 \\ qichen.dong@cern.ch \\ linkedin.com/in/qichendong} % Your phone number and email
\begin{document}

    \begin{rSection}{Education}
        \begin{rSubsection}{University of Manchester}{September 2020 - February 2025}{PhD High Energy Physics}{Manchester, UK}
            \item Thesis: \textit{Novel Boosted $\mathbf{\tau}_\mathrm{lep}\mathbf{\tau}_\mathrm{had}$ Reconstruction Techniques for 
                TeV--Scale Graviton Search in $HH\rightarrow {b}{\bar{b}}{\tau}_\mathrm{\mu}{\tau}_\mathrm{had}$ Channel with the ATLAS Detector}
        \end{rSubsection}
        \begin{rSubsection}{University of Manchester}{September 2018 - June 2020}{Master of Physics}{Manchester, UK}
            \item First Class Honours
            \item Ranked 30th out of 150 students in the final year.
        \end{rSubsection}
        \begin{rSubsection}{Shandong University}{September 2015 - June 2018}{BSc Physics}{Jinan, CN}
            \item Overall GPA: 83.3
        \end{rSubsection}
    \end{rSection}

    \begin{rSection}{Work Experience}
        \begin{rSubsection}{University of Oxford}{April 2025 - April 2028}{Postdoctoral Research Associate}{Oxford, UK}
            \item   Three-year post with a prospective start date in April 2025.
            \item   Supervised by Prof. Chris Hays. 
        \end{rSubsection}
        \begin{rSubsection}{Qube Research and Technology}{August 2022 - February 2023}{Intern Quantitative Researcher}{London, UK}
            \item Main contributor to the 10-Q and 10-K financial reports similarity analysis using NLP techniques.
            \item Conducted a similar project on sentiment analysis of Japanese financial reports.
            \item Designed and developed novel algorithms to identify abnormal data in the financial time-series.
            \item Developed and validated a price correction model for corporate events for equities.
            \item Collaborated with a team to develop and optimise database infrastructure for real-time financial market data, enhancing data retrieval and storage efficiency.
            \item Hold return offer for a full-time position.
        \end{rSubsection}
        \begin{rSubsection}{University of Manchester}{September 2020 - September 2023}{Graduate Teaching Assistant}{Manchester, UK}
            \item   Demonstrator in the C++ / Python laboratories for year-2 and year-3 undergraduate physics students.
            \item   Demonstrator in the year-3 particle physics laboratory for undergraduate students. 
            \item   Prepared a set of interactive electromagnetic field and radiation animations 
                for the year-3 electrodynamics course. Used in the lectures and made available 
                offline to the students to use and modify.
        \end{rSubsection}
    \end{rSection}

    \begin{rSection}{Research Experience}
        \begin{rSubsection}{ATLAS experiment, CERN}{April 2025 - September 2028}{ATLAS Software Development Grant}{Geneva, CH}
            \item   Six-month project to develop and enhance the ATLAS offline software.
            \item   Merge custom inference implementations from various groups into a unified version.
            \item   Integrate new machine learning inferencing technology.
        \end{rSubsection}
        \begin{rSubsection}{University of Manchester}{September 2020 - September 2024}{PhD Project}{Manchester, UK}
            \item   Supervised by Prof. Terry Wyatt FRS. 
            \item   Proposed, developed, implemented, and tested improved methods to identify the highly boosted pair production of the $\tau$ leptons in the lep-had channels --- the electron-removal $\tau_\mathrm{had}$ and the muon-removal $\tau_\mathrm{had}$ reconstruction applied in Athena.
            \item   Algorithms went through strict scrutiny, now running in Tier-0 ATLAS data processing system. These methods have been
                adopted by the ATLAS collaboration as the recommanded taggers for boosted $\tau_\mathrm{lep}\tau_\mathrm{had}$ identification.
            \item   The muon-removal $\tau_\mathrm{had}$ technique has been benchmarked with data, achieving a three-~to~five-fold 
                improvement in the signal efficiency and signal-to-background ratio. Paper accepted by EPJC.
            \item   Single-handedly performing a search for resonant production of Higgs boson pairs in the highly boosted $bb\tau\tau$ channel. 
            \item   Member of the Run 2+3 $H\rightarrow aa\rightarrow \mu\mu\tau_\mathrm{\mu}\tau_\mathrm{had}$ analysis, which uses the muon-removal $\tau_\mathrm{had}$ reconstruction as a key ingredient.
            \item   Three papers are scheduled to be published for the benefit of the ATLAS collaboration, in which I will be the primary author.
            \item   Presented the TauCP group summary talk at the ATLAS 30th birthday week, 2022, Lisbon.
            \item   Expert reviewer for the ATLAS Run 2 $H\rightarrow aa\rightarrow 4\tau$ analysis.
        \end{rSubsection}
        \begin{rSubsection}{ATLAS experiment, CERN}{April 2022 - August 2022}{Developer and Maintainer}{Geneva, CH}
            \item Long-term-attached PhD student.
            \item Developer and reviewer for Athena, the ATLAS offline software. 
            \item Senior shifter in the ATLAS software merge requests review team.
        \end{rSubsection}
    \end{rSection}


    \begin{rSection}{Additional Experience}
        \begin{rSubsection}{Gasala, Remote Village}{June 2016 - September 2017}{Volunteer Primary School Teacher}{Sichuan, CN}
            \item Led a team of volunteers.
            \item Organised math and science lessons for school-age children. 
        \end{rSubsection}
        \begin{rSubsection}{Shandong University}{September 2015 - June 2017}{Associate Lead of Student Union}{Jinan, CN}
            \item   Organised voluntary activities for the university.
            \item   Led a team of student representatives.
        \end{rSubsection}
    \end{rSection}
    
    \begin{rSection}{Skills and Interests}
        \begin{tabular}{ @{} >{\bfseries}l @{\hspace{6ex}} l }
            Programming  & Proficient in programming with C/C++ and Python \\
            Data related & Machine learning, big data analysis, statistical analysis \\
            Teamwork     & Strong communication skills in highly collaborative environments \\ 
            Languages    & Chinese (Native), English (Professional) \\ 
            Interests    & Graphic design, accelarated / distributed computing \\  
        \end{tabular}
    \end{rSection}
\end{document}