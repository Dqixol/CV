
\documentclass{resume} % Use the custom resume.cls style

\usepackage[left=0.75in,top=0.6in,right=0.75in,bottom=0.6in]{geometry} % Document margins
\usepackage{enumitem}
\newcommand{\tab}[1]{\hspace{.2667\textwidth}\rlap{#1}}
\newcommand{\itab}[1]{\hspace{0em}\rlap{#1}}
\name{Qichen Dong} % Your name
\address{207 Chapel Street, Salford, UK, M3 5PH} % Your address
\address{(+44) 742-250-7895 \\ qichen.dong@cern.ch \\ linkedin.com/in/qichendong} % Your phone number and email
\begin{document}

    \begin{rSection}{Education}
        \begin{rSubsection}{University of Manchester}{September 2020 - September 2024}{PhD Student in High Energy Physics}{Manchester, UK}
            \item Expected September 2024 
        \end{rSubsection}
        \begin{rSubsection}{University of Manchester}{September 2018 - June 2020}{Master of Physics}{Manchester, UK}
            \item First Class Honours
            \item Overall GPA: 74.4
            \item Ranked 30th out of 150 students in the final year.
        \end{rSubsection}
        \begin{rSubsection}{Shandong University}{September 2015 - June 2018}{Bachelor of Science in Physics}{Jinan, CN}
            \item Overall GPA: 83.3
        \end{rSubsection}
    \end{rSection}

    \begin{rSection}{Research Experience}
        \begin{rSubsection}{University of Manchester}{Sep 2020 - September 2024}{PhD student}{Manchester, UK}
            \item   Supervised by Prof. Terry Wyatt FRS. Thesis submitted on 30th September 2024.
            \item   Proposed, developed, implemented, and tested improved methods to identify the highly boosted pair production of the $\tau$ leptons in the lep-had channels --- the electron-removal $\tau_\mathrm{had}$ and the muon-removal $\tau_\mathrm{had}$ reconstruction.
            \item   Algorithms went through strict scrutiny, now running in Tier-0 ATLAS data processing system. These methods have been
                adopted by the ATLAS collaboration as the recommanded taggers for boosted $\tau_\mathrm{lep}\tau_\mathrm{had}$ identification.
            \item   The muon-removal $\tau_\mathrm{had}$ technique has been benchmarked with data, achieving a three-~to~four-fold 
                improvement in the signal efficiency and signal-to-background ratio. ATLAS paper~\texttt{TAUP-2023-02}~is in ATLAS collaboration circulation.
            \item   Single-handedly performing a search for resonant production of Higgs boson pairs in the highly boosted $bb\tau\tau$ channel \texttt{HDBS-2024-09}. 
            \item   Member of the Run 2+3 $H\rightarrow aa\rightarrow \mu\mu\tau_\mathrm{\mu}\tau_\mathrm{had}$ analysis, which uses the muon-removal $\tau_\mathrm{had}$ reconstruction as a key ingredient.
            \item   Three papers are scheduled to be published for the benefit of the ATLAS collaboration, in which I will be the primary author.
            \item   Presented the TauCP group summary talk at the ATLAS 30th birthday week, 2022, Lisbon.
            \item   Expert reviewer for the Run 2 ATLAS $H\rightarrow aa\rightarrow 4\tau$ analysis.
        \end{rSubsection}
        \begin{rSubsection}{ATLAS experiment, CERN}{April 2022 - August 2022}{Developer and Maintainer}{Geneva, CH}
            \item Long-term-attached PhD student.
            \item Developer and reviewer for Athena, the ATLAS offline software. Senior shifter in the ATLAS software merge-requests review team.
        \end{rSubsection}
        \begin{rSubsection}{University of Manchester}{September 2019 - June 2020}{Master Project}{Manchester, UK}
            \item   Project supervised by Prof. Andrew Pilkington. Achieved 81\%.
            \item   Searching for extra source of CP violation with the Vector Boson Scattering (VBS) processes.
            \item   One of the first to set preliminary limits on the Standard Model Effective Field Theory parameters in VBS processes.
        \end{rSubsection}
    \end{rSection}

    \begin{rSection}{Work Experience}
        \begin{rSubsection}{Qube Research and Technology}{August 2022 - February 2023}{Staffed Intern Quantitative Researcher}{London, UK}
            \item Salaried internship.
            \item Main contributor to the 10-Q and 10-K financial reports similarity analysis using NLP techniques.
            \item Conducted a similar project on sentiment analysis of Japanese financial reports.
            \item Designed and developed novel algorithms to identify abnormal data in the financial time-series.
            % \item Developed and validated a price correction model for corporate events for equities.
            % \item Collaborated with a team to develop and optimise database infrastructure for real-time financial market data, enhancing data retrieval and storage efficiency.
        \end{rSubsection}
        \begin{rSubsection}{University of Manchester}{September 2020 - September 2023}{Teaching Assistant}{Manchester, UK}
            \item   Demonstrator in the C++ / Python laboratories for year-2 and year-3 undergraduate physics students.
            \item   Demonstrator in the year-3 particle physics laboratory for undergraduate students. 
            \item   Prepared a set of interactive electromagnetic field and radiation animations 
                for the year-3 electrodynamics course. Used in the lectures and made available 
                offline to the students to use and modify.
        \end{rSubsection}
    \end{rSection}


    \begin{rSection}{Additional Experience}
        \begin{rSubsection}{Remote area in Sichuan}{June 2016 - September 2017}{Volunteer Primary School Teacher}{Sichuan, CN}
            \item   Helped primary school students who live in remote area in Sichuan province during summer break.
            \item   The team helped hundreds of children.
        \end{rSubsection}
        \begin{rSubsection}{Shandong University}{September 2015 - June 2017}{Lead of Student Union}{Jinan, CN}
            \item   Responsible for organising voluntary works hosted by the university.
            \item   Managing a team of 20 SU undergraduate representatives.
        \end{rSubsection}
    \end{rSection}
    
    \begin{rSection}{Skills and Interests}
        \begin{tabular}{ @{} >{\bfseries}l @{\hspace{6ex}} l }
            Programming & Proficient in programming with C/C++ and Python \\
            Teamwork    & Strong communication skills in highly collaborative environments \\ 
            Languages   & Chinese (Native), English (Professional) \\ 
            Interests   & Graphic design, accelarated / distributed computing \\  
        \end{tabular}
    \end{rSection}
\end{document}