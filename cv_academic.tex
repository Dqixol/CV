
\documentclass[11pt]{resume} % Use the custom resume.cls style

\usepackage[left=2cm,top=2cm,right=2cm,bottom=2cm]{geometry} % Document margins
\usepackage{enumitem}
\usepackage{helvet} % Helvetica font (closest to Arial)
\renewcommand{\familydefault}{\sfdefault} % Set sans-serif as default font family
\newcommand{\tab}[1]{\hspace{.2667\textwidth}\rlap{#1}}
\newcommand{\itab}[1]{\hspace{0em}\rlap{#1}}
\name{Qichen Dong} % Your name
\address{CERN, Geneva, Switzerland} % Your address
\address{(+44) 742-250-7895 \\ qichen.dong@cern.ch \\ linkedin.com/in/qichendong} % Your phone number and email

\usepackage{hyperref}
\hypersetup{
    colorlinks=true,
    urlcolor=blue,
}
\usepackage{biblatex}
\usepackage{atlasbiblatex}
\addbibresource{cv_academic.bib}

\begin{document}
    \nocite{*}

    \begin{rSection}{Work Experience}
        \begin{rSubsection}{University of Oxford}{April 2025 - Present}{Postdoctoral Research Associate}{Geneva, CH}
            \item Supervised by Prof. Chris Hays; leading development of the ATLAS boosted di-\(\tau_{\mathrm{had}}\) trigger for the 2025 LHC run.
            \item Designed and deployed real-time di-\(\tau_{\mathrm{had}}\) trigger algorithms, enhancing signal selection efficiency and background estimation.
            \item Recipient of an ATLAS Software Development Grant to enhance inference performance of existing machine-learning models within trigger and reconstruction frameworks.
            \item Key analyst on the ATLAS Run 2+3 boosted \(H\to\tau_{\mathrm{had}}\tau_{\mathrm{had}}\) search: trigger optimisation, signal extraction, and BSM interpretation.
            \item Developer of the lepton-removal \(\tau_{\mathrm{had}}\) reconstruction for the \(H\to aa\to\mu\mu\tau_{\mu}\tau_{\mathrm{had}}\) analysis, including calibration and performance validation.
            \item Co-supervise two summer students: one developing trigger monitoring algorithms and one supporting the interpretation of the boosted \(H\to\tau_{\mathrm{had}}\tau_{\mathrm{had}}\) analysis.
        \end{rSubsection}
        \begin{rSubsection}{Qube Research and Technology}{Aug 2022 - Feb 2023}{Intern Quantitative Researcher}{London, UK}
            \item Applied natural language processing (NLP) models for company similarity analysis on 10-Q/10-K filings and performed sentiment analysis on Japanese local financial reports.
            \item Developed ML algorithms for financial time-series anomaly detection and corporate-event price correction; optimised real-time market data pipelines.
            \item Secured a return offer for a permanent quantitative researcher position.
        \end{rSubsection}
        \begin{rSubsection}{University of Manchester}{Sep 2020 - Sep 2023}{Graduate Teaching Assistant}{Manchester, UK}
            \item Assisted C++ and Python lab sessions for 2nd- and 3rd-year physics undergraduates.
            \item Supported 3rd-year particle physics experiments, guiding students through data analysis.
            \item Created interactive Python animations of electromagnetic fields and radiation for the electrodynamics course--used in lectures and distributed for student customisation.
        \end{rSubsection}
    \end{rSection}

    \begin{rSection}{Education}
        \begin{rSubsection}{University of Manchester}{September 2020 - February 2025}{PhD in Particle Physics}{Manchester, UK}
            \item Earned one first-author publication with several more in preparation.
        \end{rSubsection}
        \begin{rSubsection}{University of Manchester}{September 2018 - June 2020}{Master of Physics}{Manchester, UK}
            \item First-Class Honours; Ranked 30th out of 150 in the final year.
        \end{rSubsection}
        \begin{rSubsection}{Shandong University}{September 2015 - June 2018}{BSc Physics}{Jinan, CN}
            \item GPA: 83.3; core physics courses GPA: 88.6.
        \end{rSubsection}
    \end{rSection}

    \begin{rSection}{Research Experience}
        \begin{rSubsection}{ATLAS experiment, CERN}{May 2025 - Present}{ATLAS Software Development Grant--AI Inference}{Geneva, CH}
            \item Spearheading a six-month AI initiative to optimise ATLAS ML inference pipelines.
            \item Consolidating and standardising diverse ML models from multiple sub-groups into a unified Athena (the ATLAS offline software framework) inference framework.
            \item Exploring initial integration of NVIDIA Triton inference server within Athena to assess potential improvements in inference throughput and scalability.
        \end{rSubsection}
        \begin{rSubsection}{University of Manchester}{September 2020 - September 2024}{PhD Projects}{Manchester, UK}
            \item   Supervised by Prof. Terry Wyatt FRS. 
            \item   Proposed, developed, implemented, and tested improved methods to identify the highly boosted pair production of the $\tau$ leptons in the lep-had channels---the electron-removal $\tau_\mathrm{had}$ and the muon-removal $\tau_\mathrm{had}$ reconstruction applied in Athena.
            \item   Algorithms went through strict scrutiny, now running in Tier-0 ATLAS data processing system. These methods have been
                adopted by the ATLAS collaboration as the recommanded taggers for boosted $\tau_\mathrm{lep}\tau_\mathrm{had}$ identification.
            \item   The muon-removal $\tau_\mathrm{had}$ technique has been benchmarked with data, achieving a three-~to~five-fold 
                improvement in the signal efficiency and signal-to-background ratio. Paper published by EPJC.
            \item   Single-handedly performed a search for resonant production of Higgs boson pairs in the highly boosted $bb\tau\tau$ channel. 
            \item   Member of the Run 2+3 $H\rightarrow aa\rightarrow \mu\mu\tau_\mathrm{\mu}\tau_\mathrm{had}$ analysis, which uses the lepton-removal $\tau_\mathrm{had}$ reconstruction as a key ingredient.
            \item   Two papers based on my PhD research are scheduled for publication with the ATLAS Collaboration; I am the primary author.
            \item   Presented the TauCP group summary talk at the ATLAS 30th birthday week, 2022, Lisbon.
            \item   Expert reviewer for the ATLAS Run 2 $H\rightarrow aa\rightarrow 4\tau$ analysis.
        \end{rSubsection}
        \begin{rSubsection}{ATLAS experiment, CERN}{April 2022 - August 2022}{Long-term-attached PhD student.}{Geneva, CH}
            \item Developer and reviewer for Athena, the ATLAS offline software. 
            \item Senior shifter in the ATLAS software merge requests review team.
        \end{rSubsection}
    \end{rSection}

    \begin{rSection}{Volunteer \& Leadership Experience}
        \begin{rSubsection}{Gasala, Remote Village}{June 2016 - September 2017}{Volunteer Primary School Teacher}{Sichuan, CN}
            \item Led a team of volunteers.
            \item Organised math and science lessons for school-age children. 
        \end{rSubsection}
        \begin{rSubsection}{Shandong University}{September 2015 - June 2017}{Associate Lead of Student Union}{Jinan, CN}
            \item   Organised voluntary activities for the university.
            \item   Led a team of student representatives.
        \end{rSubsection}
    \end{rSection}
    
    \begin{rSection}{Skills \& Interests}
        \begin{tabular}{ @{} >{\bfseries}l @{\hspace{6ex}} l }
            Programming  & Proficient in programming with C/C++ and Python \\
            Data related & Machine learning, big data analysis, statistical analysis \\
            Teamwork     & Strong communication skills in highly collaborative environments \\ 
            Languages    & Chinese (Native), English (Full Professional Proficiency) \\ 
            Interests    & Graphic design, accelarated / distributed computing \\  
        \end{tabular}
    \end{rSection}

    \begin{rSection}{Publications}
        As a member of the ATLAS Collaboration, I have co-authored over 300 peer-reviewed articles since 2022, 
        contributing to all stages of the experimental program—from advanced data analysis techniques and detector 
        performance studies to the development and optimization of cutting-edge methodologies. 
        My efforts have been integral to several high-impact results, enhancing the collaboration's 
        scientific output. A complete list of my publications is available on \href{https://inspirehep.net/authors/2164455}{InspireHEP}.

        \textbf{Publications (Major Contributions)}
            \printbibliography[heading=none,type=article,title={ATLAS Publications (Major Contributions)}]
        \textbf{PhD Thesis}
            \printbibliography[heading=none,type=thesis,title={PhD Thesis}]
        
    \end{rSection}
\end{document}